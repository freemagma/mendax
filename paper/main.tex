\documentclass[10pt,twocolumn,letterpaper]{article}

\usepackage{cvpr}
\usepackage{times}
\usepackage{epsfig}
\usepackage{graphicx}
\usepackage{amsmath}
\usepackage{amssymb}

% Include other packages here, before hyperref.

% If you comment hyperref and then uncomment it, you should delete
% egpaper.aux before re-running latex.  (Or just hit 'q' on the first latex
% run, let it finish, and you should be clear).
\usepackage[breaklinks=true,bookmarks=false]{hyperref}

\cvprfinalcopy % *** Uncomment this line for the final submission

\def\cvprPaperID{****} % *** Enter the CVPR Paper ID here
\def\httilde{\mbox{\tt\raisebox{-.5ex}{\symbol{126}}}}

% Pages are numbered in submission mode, and unnumbered in camera-ready
%\ifcvprfinal\pagestyle{empty}\fi
\setcounter{page}{4321}
\begin{document}

%%%%%%%%% TITLE
\title{Machines are Among Us}

\author{Zach Minot\\
Georgia Institute of Technology\\
2nd Year Undergraduate, CS\\
{\tt\small zjminot@gatech.edu}
% For a paper whose authors are all at the same institution,
% omit the following lines up until the closing ``}''.
% Additional authors and addresses can be added with ``\and'',
% just like the second author.
% To save space, use either the email address or home page, not both
\and
Charles Gunn\\
Georgia Institute of Technology\\
2nd Year Undergraduate, CS \& Math\\
{\tt\small cgunn30@gatech.edu}
}

\maketitle
%\thispagestyle{empty}

%%%%%%%%% ABSTRACTcommunicate
\begin{abstract}
   The ABSTRACT is to be in fully-justified italicized text, at the top
   of the left-hand column, below the author and affiliation
   information. Use the word ``Abstract'' as the title, in 12-point
   Times, boldface type, centered relative to the column, initially
   capitalized. The abstract is to be in 10-point, single-spaced type.
   Leave two blank lines after the Abstract, then begin the main text.
   Look at previous CVPR abstracts to get a feel for style and length.
\end{abstract}

%%%%%%%%% BODY TEXT
\section{Introduction}

To what extent can neural network models communicate with each other and discover
each other's identity? How would they use this information in a competitive setting?
For example, in a social deduction game, players attempt to uncover each other's
hidden allegiance---typically with one "good" team and one "bad" team.
Players must utilize deductive reasoning to find the truth or instead
lie to keep their role hidden. In this paper, we explore 
if neural networks can be successfully trained to
compete in a scenario such as this, and how would the opposing parties interact
during the period of debate.

\subsection{Among Us}
Among Us is a currently popular social deduction game,
where the "imposters" attempt to sabotage and kill all of the "crewmates". Crewmmates have
to complete tasks and figure out who the imposters are and eliminate them before the imposters
win.
At certain points in the game, after periods of no direct communication, players 
debate the roles of each individual based on information previously acquired through
their personal experience. At the end of this discussion, every player votes on a single
player to be eliminated. The player with the most votes is eliminated, and if there
is a tie, no one is voted out. We chose to emulate this game based on
the overall simplicity of the two roles and the requirement of communication
for either party to succeed. If the crew do not exchange information 
and all vote the same person, 
the vote could result in a tie or a crew being eliminated.
If the imposters do not bluff, the crew can easily spot the liars among the group. 
This provides ample room to explore and experiment with the communication between
the two opposing parties.
\subsection{Adverserial Networks}
Within this design space, there are adversarial parties working against each other.
In the deep learning realm, adversarial situations appear in adversarial examples~\cite{AdverserialEx} and within GANs (generative adversarial networks)
~\cite{NIPS2014_5ca3e9b1}
In particular, the latter often designs a contest between two neural networks, in the form
of a zero-sum game.
We build upon these concepts and foundations in our work.
\subsection{Multi-agent Communication}
Inherently, a social deduction game requires multiple agents to be trained and contested.
This has been explored within the deep learning problem space with multi-agent subproblems.
Both cooperative~\cite{DBLP:journals/corr/FoersterAFW16}~\cite{DBLP:journals/corr/FoersterAFW16a} 
and adversarial ~\cite{DBLP:journals/corr/AbadiA16} communication has been experimented with,
showing that models can effective share and also selectively protect information.
We reference these approaches we generate active adversarial communication between 
neural network models.
\section{Approach}
\subsection{Deductive Situation}
\subsection{Modeling Interpretation and Communication}
\subsection{Zero-sum Target}
\subsection{Training Scheme}
\subsection{Challenges}

\section{Results}
% talk about our goal for a qualitatively "good" communication
% session.
\subsection{Oscillating Scores}
% talk about problems we saw frequently. show examples of "good" runs
% and bad runs
\subsection{Situation Hyperparameters}
% grid search results.
\subsection{Model Evolution}
% are the agents "improving" or just changing minorly to trick each other
\subsection{Interpreting Communcation Vectors}
\subsection{Future Directions}


{\small
\bibliographystyle{ieee_fullname}
\bibliography{egbib}
}

\end{document}
