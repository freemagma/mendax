\documentclass[10pt,twocolumn,letterpaper]{article}

\usepackage{cvpr}
\usepackage{times}
\usepackage{epsfig}
\usepackage{graphicx}
\usepackage{amsmath}
\usepackage{amssymb}

% Include other packages here, before hyperref.

% If you comment hyperref and then uncomment it, you should delete
% egpaper.aux before re-running latex.  (Or just hit 'q' on the first latex
% run, let it finish, and you should be clear).
\usepackage[breaklinks=true,bookmarks=false]{hyperref}

\cvprfinalcopy % *** Uncomment this line for the final submission

\def\cvprPaperID{****} % *** Enter the CVPR Paper ID here
\def\httilde{\mbox{\tt\raisebox{-.5ex}{\symbol{126}}}}

% Pages are numbered in submission mode, and unnumbered in camera-ready
%\ifcvprfinal\pagestyle{empty}\fi
\setcounter{page}{4321}
\begin{document}

%%%%%%%%% TITLE
\title{Machines are Among Us}

\author{Zach Minot\\
Georgia Institute of Technology\\
2nd Year Undergraduate, CS\\
{\tt\small zjminot@gatech.edu}
% For a paper whose authors are all at the same institution,
% omit the following lines up until the closing ``}''.
% Additional authors and addresses can be added with ``\and'',
% just like the second author.
% To save space, use either the email address or home page, not both
\and
Charles Gunn\\
Georgia Institute of Technology\\
2nd Year Undergraduate, CS \& Math\\
{\tt\small cgunn30@gatech.edu}
}

\maketitle
%\thispagestyle{empty}

%%%%%%%%% ABSTRACT
\begin{abstract}
   The ABSTRACT is to be in fully-justified italicized text, at the top
   of the left-hand column, below the author and affiliation
   information. Use the word ``Abstract'' as the title, in 12-point
   Times, boldface type, centered relative to the column, initially
   capitalized. The abstract is to be in 10-point, single-spaced type.
   Leave two blank lines after the Abstract, then begin the main text.
   Look at previous CVPR abstracts to get a feel for style and length.
\end{abstract}

%%%%%%%%% BODY TEXT
\section{Introduction}
\subsection{Adverserial Networks}
\subsection{Multi-agent Communication}

\section{Approach}
\subsection{Deductive Situation}
\subsection{Modeling Interpretation and Communication}
\subsection{Zero-sum Target}
\subsection{Training Scheme}
\subsection{Challenges}

\section{Results}
% talk about our goal for a qualitatively "good" communication
% session.
\subsection{Oscillating Scores}
% talk about problems we saw frequently. show examples of "good" runs
% and bad runs
\subsection{Situation Hyperparameters}
% grid search results.
\subsection{Model Evolution}
% are the agents "improving" or just changing minorly to trick each other
\subsection{Interpreting Communcation Vectors}
\subsection{Future Directions}


{\small
\bibliographystyle{ieee_fullname}
\bibliography{egbib}
}

\end{document}
